
\input{style/short_commands_malte}

\documentclass[11pt]{beamer}	% Available font sizes 8pt, 9pt, 10pt, 11pt, 12pt, 14pt, 17pt, 20pt. Default font size 11pt (corresponds to 22pt at the full screen mode)
\usepackage[utf8]{inputenc}

\usetheme{Darmstadt}
\usecolortheme{beaver}
\setbeamercolor{title}{fg=blue}


% \setbeamertemplate{footline}[frame number]
\setbeamertemplate{footline}{\hfill\strut\insertframenumber\quad\quad}
\setbeamertemplate{navigation symbols}{\insertslidenavigationsymbol}

\usepackage{natbib}	%defines citation style
% \citestyle{aa}


% title slide
\title[title in foot]{Defense}
\subtitle{Evidence from Space}
\author[MV]{Malte Venzmer}
\institute[IAG]{
	Institute for Astrophysics\\
	Georg-August Universität Göttingen
}
% \date{\today}
\date[KPT 2004]{Defense, \today}
\subject{subject: solar wind}
% \logo{\includegraphics[height=1cm]{images/IAG_LogoRGB.png}}


\begin{document}

\frame{\titlepage}

\AtBeginSection[]{
	\subsection{}	% for making the nav dots appear
	\frame{\tableofcontents[currentsection]}
}

\section{Abschnitt 1}

\begin{frame}[c]{first slide}{A bit more information about this}
	This is a text in first frame. \pause This is a text in first frame. This is a text in first frame.
	\begin{definition}
		A definition
	\end{definition}
	\url{https://www.sharelatex.com/learn/latex/Beamer}
\end{frame}

\section{Section 2}

\begin{frame}[<+->]{Ein Demotitel}{}
	\begin{itemize}
		\item<1-> dasgfs
		\item<2> asgag
		\item<3-> asgag
		\item asgag
	\end{itemize}
\end{frame}

\begin{frame}[<+->]{Ein Demotitel}{}
	\begin{itemize}
		\item dasgfs
		\item asgag
		\item asgag
		\item<.-> asgag
	\end{itemize}
\end{frame}

\begin{frame}{Ein Demotitel}{}
	\begin{itemize}
		\item<+-> dasgfs
		\item<+-> asgag
		\item<+-> asgag
		\item<+-> asgag
	\end{itemize}
\end{frame}

\section{part two}

\begin{frame}[t]{Sample frame title}{}
	
	In this slide, some important text will be
	\alert<2->{highlighted} beause it's important.
	Please, don't abuse it.
	
	\onslide<2>{	%\uncover{}
	\begin{block}{Remark}
	Sample text
	\end{block}
	}
	
	\onslide*<1>{	%\only{}
	\begin{examples}
	Sample text in green box. "Examples" is fixed as block title.
	\end{examples}
	}
	
	\onslide+<2>{	%\visible{}
	\begin{alertblock}{Important theorem}
	Sample text in red box
	\end{alertblock}
	}
	
\end{frame}

\begin{frame}[plain]{Sample frame title}{}
	
	In this slide, some important text will be
	\alert{highlighted} beause it's important.
	Please, don't abuse it.
	
	\begin{block}{Remark}
	Sample text
	\end{block}
	
	\begin{alertblock}{Important theorem}
	Sample text in red box
	\end{alertblock}
	
	\begin{examples}
	Sample text in green box. "Examples" is fixed as block title.
	\end{examples}
\end{frame}

\section{part 3}

\begin{frame}{Two-column slide}{}
	\begin{columns}[t]
		\column{0.5\textwidth}
		This is a text in first column.
		$$E=mc^2$$
		\begin{itemize}
		\item First item
		\item Second item
		\end{itemize}
		
		\column{0.5\textwidth}
		This text will be in the second column
		and on a second tought this is a nice looking
		layout in some cases \citep{Venzmer2018}.
	\end{columns}
\end{frame}

\begin{frame}[t]{Sample}{}
	
	In this slide, some important text will be
	\alert<2->{highlighted} beause it's important.
	Please, don't abuse it.
	
	\begin{block}<1->{Remark}
	Sample text
	\end{block}
	
	\begin{examples}<2>
	Sample text in green box. "Examples" is fixed as block title.
	\end{examples}
	
	\begin{alertblock}<3>{Important theorem}
	Sample text in red box
	\end{alertblock}
	
\end{frame}

\begin{frame}[allowframebreaks]{Further Reading}
	\begin{thebibliography}{10}
	
		\beamertemplatebookbibitems
		
		\beamertemplatearticlebibitems
		
		\bibitem[{{Parker}(1958)}]{Parker1958}
			{Parker}, E.~N. 1958, \emph{{Dynamics of the Interplanetary Gas and Magnetic Fields.}}, \apj, 128, 664, \href{http://dx.doi.org/10.1086/146579}{[DOI]}, \href{http://adsabs.harvard.edu/abs/1958ApJ...128..664P}{[ADS]}.
			
		\bibitem[{{Venzmer} \& {Bothmer}(2018)}]{Venzmer2018}
			{Venzmer}, M.~S. \& {Bothmer}, V. 2018, \emph{{Solar-wind predictions for the
			Parker Solar Probe orbit. Near-Sun extrapolations derived from an empirical
			solar-wind model based on Helios and OMNI observations}}, \aap, 611, A36,
			\href{http://dx.doi.org/10.1051/0004-6361/201731831}{[DOI]},
			\href{http://adsabs.harvard.edu/abs/2018A\%26A...611A..36V}{[ADS]}.

	\end{thebibliography}
\end{frame}

\appendix

\section{Backup slides}

\begin{frame}{backup slide}{}
	gj
\end{frame}

\section{Backup slides 2}

\begin{frame}{backup slide}{}
	gj
\end{frame}

% \begin{frame}[]{title}{subtitle}
% 
% \end{frame}


\end{document}


